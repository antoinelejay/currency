% \iffalse meta-comment
%
% Copyright (C) 2015 by Scott Pakin <scott+dtx@pakin.org>
% -------------------------------------------------------
% 
% This file may be distributed and/or modified under the
% conditions of the LaTeX Project Public License, either version 1.3
% of this license or (at your option) any later version.
% The latest version of this license is in:
%
%    http://www.latex-project.org/lppl.txt
%
% and version 1.3 or later is part of all distributions of LaTeX 
% version 2005/12/01 or later.
%
% \fi
%
% \iffalse
%<*driver>
\ProvidesFile{currency.dtx}
%</driver>
%<package>\NeedsTeXFormat{LaTeX2e}[2005/12/01]
%<package>\ProvidesPackage{currency}
%<*package>
    [2018/10/22 v0.4 .dtx currency file]
%</package>
%
%<*driver>
\documentclass{ltxdoc}
\usepackage{currency}
\usepackage{url}
\EnableCrossrefs         
\CodelineIndex
\RecordChanges
\begin{document}
  \DocInput{currency.dtx}
  \PrintChanges
  \PrintIndex
\end{document}
%</driver>
% \fi
%
% \CheckSum{0}
%
% \CharacterTable
%  {Upper-case    \A\B\C\D\E\F\G\H\I\J\K\L\M\N\O\P\Q\R\S\T\U\V\W\X\Y\Z
%   Lower-case    \a\b\c\d\e\f\g\h\i\j\k\l\m\n\o\p\q\r\s\t\u\v\w\x\y\z
%   Digits        \0\1\2\3\4\5\6\7\8\9
%   Exclamation   \!     Double quote  \"     Hash (number) \#
%   Dollar        \$     Percent       \%     Ampersand     \&
%   Acute accent  \'     Left paren    \(     Right paren   \)
%   Asterisk      \*     Plus          \+     Comma         \,
%   Minus         \-     Point         \.     Solidus       \/
%   Colon         \:     Semicolon     \;     Less than     \<
%   Equals        \=     Greater than  \>     Question mark \?
%   Commercial at \@     Left bracket  \[     Backslash     \\
%   Right bracket \]     Circumflex    \^     Underscore    \_
%   Grave accent  \`     Left brace    \{     Vertical bar  \|
%   Right brace   \}     Tilde         \~}
%
%
% \changes{v0.1}{2016/05/25}{Initial version}
% \changes{v0.2}{2016/06/15}{Adding vXXX as an alias for dXXX}
% \changes{v0.3}{2017/07/04}{Dropping spurious space in the definition of cXXX}
% \changes{v0.4}{2018/10/22}{Change DeclareDocumentCommand for compability with TexLive 2018}
%
%
% \GetFileInfo{currency.dtx}
%
% \DoNotIndex{\newcommand,\newenvironment}
% 
%
% \title{The \textsf{currency} package\thanks{This document
%   corresponds to \textsf{currency}~\fileversion, dated \filedate.}}
% \author{Antoine Lejay \\ \texttt{Antoine.Lejay@univ-lorraine.fr}}
%
% \maketitle
%
% 
% \DefineCurrency{EUR}{name={euro},plural-name={euros},symbol={\euro},iso={EUR},kind=iso}
% \DefineCurrency{USD}{name={dollar},plural-name={dollars},symbol={\$},iso={USD},kind=iso}
%
% \section{Introduction}
%
% This package creates macros for defined currencies which follow
% the ISO 4217 codes, with various formatting options for both
% the currency (code, symbols, names, ...) and the numbers (using \texttt{siunitx}).
% 
% The currency code ISO 4217 specifies the code of the currency as
% a three-letters code. The first two ones are the code of the country
% according to ISO 3166. The last one is the name of the currency name. 
%
% \section{Licence}
%
% This work may be distributed and/or modified under the
% conditions of the LaTeX Project Public License, either version 1.3
% of this license or (at your option) any later version.
% The latest version of this license is in
%   \url{http://www.latex-project.org/lppl.txt}
% and version 1.3 or later is part of all distributions of LaTeX
% version 2005/12/01 or later.
%
% This work has the LPPL maintenance status `maintained'.
%
% The Current Maintainer of this work is Antoine Lejay.
%
% \section{Documentation and sources}
%
% This package is documented in \texttt{currency\_doc}.
%
% A source of this package is hosted in
% \begin{center}
% \url{https://github.com/antoinelejay/currency}
% \end{center}
% 
%
%
% \section{Implementation}
%
% \subsection{Dependencies}
%
%    \begin{macrocode}
\RequirePackage{siunitx}
\RequirePackage{pgfkeys}
\RequirePackage{etoolbox}
\RequirePackage{xparse}
\RequirePackage{expl3}
\RequirePackage{textcomp}
\RequirePackage{eurosym}
%    \end{macrocode}
% The unit could be printed before or after
%    \begin{macrocode}
\newif\ifcurrencynumber 
\pgfkeys{/currency/.cd,number/.is if=currencynumber,number=true}
\newif\ifprintbefore
\pgfkeys{/currency/pre/.is if=printbefore}
\pgfkeys{/currency/name/.initial=ZZZ}
\pgfkeys{/currency/symbol/.initial=\textcurrency}
%    \end{macrocode}
%
% Defines a style currency, which is general and empty. 
% The styles are then called in order 
%   - the currency style
%   - currency (initially empty)
%   - the options passed to the command

%    \begin{macrocode}
\pgfkeys{/currency/currency/.style={}}
%    \end{macrocode}
%
%    \begin{macrocode}
\newcommand{\CurrencySetup}[1]{%
    \pgfkeys{/currency/currency/.style={#1}}%
}
\newcommand{\CurrencySetupAppend}[1]{%
    \pgfkeys{/currency/currency/.append style={#1}}%
}
%    \end{macrocode}
%
% \subsection{Names and symbols definitions}
%
% The commands \verb|\currencyName|, \verb|\currencyNames|, \verb|\currencyISO| and \verb|\currencySymbol|
% are used internally every time a currency is called to define the name, the plural, 
% the ISO code and the symbol associated to a currency.
%
%    \begin{macrocode}
\pgfkeys{/currency/name/.store in=\currencyName}
\pgfkeys{/currency/plural/.store in=\currencyNames}
\pgfkeys{/currency/plural/.initial=\currencyName s}
\pgfkeys{/currency/iso/.store in=\currencyISO}
\pgfkeys{/currency/symbol/.store in=\currencySymbol}
%    \end{macrocode}
%
% \subsection{Keys for formatting}
%
% Several keys are devoted to insert tokens before or after th unit 
% and the amount, set up the font, ...
%
%    \begin{macrocode}
% between the symbol and the num if before
\pgfkeys{/currency/pre-between/.initial={~}} 
% between the symbol and the num if after
\pgfkeys{/currency/post-between/.initial={~}} 
% tokens that are inserted before
\pgfkeys{/currency/before/.initial={}} 
% append to before
\pgfkeys{/currency/before+/.code={\pgfkeys{/currency/before/.append={#1}}}} 
% prepend to before
\pgfkeys{/currency/before</.code={\pgfkeys{/currency/before/.prefix={#1}}}} 
\pgfkeys{/currency/after/.initial={}} % tokens that are inserted after
% append to after
\pgfkeys{/currency/after+/.code={\pgfkeys{/currency/after/.append={#1}}}} 
% prepend to after
\pgfkeys{/currency/after</.code={\pgfkeys{/currency/after/.prefix={#1}}}} 
\pgfkeys{/currency/font/.initial={}} % for setting up the fonts
% append to the font definition
\pgfkeys{/currency/font+/.code={\pgfkeys{/currency/font/.append={#1}}}} 
% the prefix before the name
\pgfkeys{/currency/prefix/.initial={}} 
%    \end{macrocode}
%
% Several options are given for printing the kind of unit (\verb|iso|, 
% \verb|plural|, \verb|name|, \verb|symbol|).
%    \begin{macrocode}
%
\pgfkeys{/currency/kind/.is choice}
\pgfkeys{/currency/kind/iso/.code=%
    {\def\currencyunit{\currencyISO}}}
\pgfkeys{/currency/kind/plural/.code=%
    {\def\currencyunit{\currencyNames}\printbeforefalse}}
\pgfkeys{/currency/kind/name/.code=%
    {\def\currencyunit{\currencyName}\printbeforefalse}}
\pgfkeys{/currency/kind/symbol/.code=%
    {\def\currencyunit{\currencySymbol}}}
\pgfkeys{/currency/kind=iso}
%    \end{macrocode}
%
% Options are also define for printing the cents (\verb|true|, \verb|false|, \verb|always|).
%
%    \begin{macrocode}
\pgfkeys{/currency/cents/.is choice}
\pgfkeys{/currency/cents/true/.code=%
    {\sisetup{zero-decimal-to-integer=true}}}
\pgfkeys{/currency/cents/false/.code=%
    {\sisetup{zero-decimal-to-integer=true,round-precision=0}}}
\pgfkeys{/currency/cents/always/.code=%
    {\sisetup{zero-decimal-to-integer=false,round-integer-to-decimal=true}}}
\pgfkeys{/currency/cents=true}
%    \end{macrocode}
%
% \subsection{Dealing with unknown keys}
%
% The unknown keys are passed to the \verb|\num| command of \texttt{siunitx}, which use the \LaTeX3 syntax
% and not the pgf one. 
%
%    \begin{macrocode}
\ExplSyntaxOn
\pgfkeys{/currency/.unknown/.code={%
\edef\Unknown{\pgfkeyscurrentname=\pgfkeyscurrentvalue}%
\keys_set:nV {siunitx} {\Unknown}%
    }}%
\ExplSyntaxOff
%    \end{macrocode}
%
% \subsection{Setting the precision}
%
% The key \verb|base| is used to set the precision. 
% Its value is passed to the key \verb|round-precision| of \verb|siunitx|.
%
%    \begin{macrocode}
\ExplSyntaxOn
\pgfkeys{/currency/base/.code={%
\edef\SetPrecision{round-precision=\pgfkeyscurrentvalue}%
\keys_set:nV {siunitx} {\SetPrecision}%
    }}%
\ExplSyntaxOff
%    \end{macrocode}
%
% \subsection{Defining new currencies}
%
% The first argument is the name of the currency, normally its ISO symbol (3 letters).
% This commands defines two global commands.
%
% \begin{macro}{\displayCurrency} 
%
% This is a generic macro for displaying the currency symbol and amount. This macro does not aim
% at being called directly.
%
%
%    \begin{macrocode}
\ExplSyntaxOn
\DeclareDocumentCommand{\displayCurrency}{m o m}{
\begingroup%
% default parameters, to put outside
\sisetup{round-mode=places,round-precision=2}
\IfNoValueTF{#2}%
    {\pgfkeys{/currency/.cd,#1,currency}}%
    {\pgfkeys{/currency/.cd,#1,currency,#2}}%
\ifbool{currencynumber}%
    {\sisetup{parse-numbers=true}}%
    {\sisetup{parse-numbers=false}}%
\edef\value{#3}% store the value of the currency
%
\ifprintbefore%
% true clause / print the unit before
\pgfkeys{/currency/font}%
\pgfkeys{/currency/before}%
\text{\pgfkeys{/currency/prefix}\currencyunit}%
\pgfkeys{/currency/pre-between}%
\num{\value}\pgfkeys{/currency/after}%
\else% print the unit after 
\pgfkeys{/currency/font}%
\pgfkeys{/currency/before}%
\num{\value}%
\pgfkeys{/currency/post-between}%
\text{\pgfkeys{/currency/prefix}\currencyunit}%
\pgfkeys{/currency/after}%
\fi%
\endgroup%
}
%    \end{macrocode}
% \end{macro}
%
% \begin{macro}{\displayCurrencySymbol} 
%
% This is a generic macro for displaying the currency symbol. This macro does not aim
% at being called directly.
%
%    \begin{macrocode}
\DeclareDocumentCommand{\displayCurrencySymbol}{m o}{
\begingroup%
    \IfNoValueTF{#2}%
	{\pgfkeys{/currency/.cd,#1,currency}}%
	{\pgfkeys{/currency/.cd,#1,currency,#2}}%
    \pgfkeys{/currency/font}%
    \pgfkeys{/currency/before}%
    \text{\pgfkeys{/currency/prefix}\currencyunit}% 
	\pgfkeys{/currency/after}%
\endgroup%
}
%    \end{macrocode}
% \end{macro}
\ExplSyntaxOff
%
% \begin{macro}{\DefineCurrency} 
% Define the parameters for a new currency. The first parameter is 
% a three letters code of type \verb|XXX| (actually, it should by any suitable word).
% Global commands \verb|\dXXX|
% (for printing currency with an amount)
% and \verb|\cXXX| (for printing only the symbols) are then defined, as
% well as a style under the pgfkey \verb|/currency/XXX|.
%
%    \begin{macrocode}
\newcommand{\DefineCurrency}[2]{%
\pgfkeys{/currency/#1/.style={#2}}%%
% define the currency with digits as \dXXX
\csgdef{d#1}{\displayCurrency{#1}}
% end of \dXXX definition
% define the currency symbol as \cXXX with one optional argument
% \global\expandafter\DeclareDocumentCommand\csname c#1\endcsname{o}{%
\csgdef{c#1}{\displayCurrencySymbol{#1}}
% end of \cXXX definition
}%
%    \end{macrocode}
% \end{macro}
%
% \subsection{Predefined styles}
%
% Some styles are defined to shorten the typesetting.
%
%    \begin{macrocode}
\pgfkeys{/currency/@iso/.style=}
\pgfkeys{/currency/@symb/.style=}
\pgfkeys{/currency/@sy/.style=}
\pgfkeys{/currency/@name/.style=}
\pgfkeys{/currency/@na/.style=}
\pgfkeys{/currency/@plural/.style=}
\pgfkeys{/currency/@pl/.style=}
\pgfkeys{/currency/no-parse/.style={number=false}}
%    \end{macrocode}
%
% \Finale
\endinput
